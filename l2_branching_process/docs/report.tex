\section*{Цель}

Исследование характеристик ветвящихся процессов для различных коэффициентов производящей функции.

\section*{Порядок выполнения}

\begin{enumerate}
    \item Для ветвящегося процесса, заданного производящей функцией, все коэффициенты которой числа строго больше нуля и меньше единицы, выбрать распределения вероятностей, при которых процесс является докритическим, критическим и надкритическим.
    \item Для каждого распределения вычислить математическое ожидание и дисперсию для величины $X_1$, вероятность вырождения процесса.
    \item Для каждого распределения осуществить моделирование процесса и экспериментально определить вероятности $P(X_n=k), k = 0,1,\dots,8$ при $n\rightarrow\infty$, для каждого значения рассчитать доверительный интервал на уровне значимости 5\%.
\end{enumerate}

\section*{Исходные данные}

Задана производящая функция

\begin{align*}
    \phi(s) = p_0 + p_2s^2 + p_3s^3 + p_4s^4 + p_5s^5 + p_6s^6.
\end{align*}

В качестве начальных значений линейного конгруентного генератора используются следующие величины:

\begin{align*}
    m & = 2^{31} - 1 & a & = 630360016 & e_0 & = 42352531
\end{align*}

\section*{Распределения вероятностей}

Первым шагом необходимо вычислить формулу математического ожидания величины $\xi$

\begin{align*}
    m = M\xi = \phi^{\prime}(1) = 2p_2 + 3p_3 + 4p_4 + 5p_5 + 6p_6.
\end{align*}

После того как определена формула математического ожидания величины $\xi$, становится возможным подбор распределений
вероятности для каждого из типов процессов на основании следующих условий:

\begin{enumerate}
    \item $m < 1$ для докритического процесса,
    \item $m = 1$ для критического процесса,
    \item $m > 1$ для надкритического процесса.
\end{enumerate}

\subsection*{Докритический процесс}

Условием докритического процесса является отношение $m < 1$, которому соответствует, например, следующее распределение вероятности:
\begin{align*}
    p_0 & = 0.85 & p_2 & = 0.05 & p_3 & = 0.04 & p_4 & = 0.03 & p_5 & = 0.02 & p_6 & = 0.01
\end{align*}

Параметры приведенного выше распределения следующие:
\begin{align*}
    m & = M\xi = 0.5 & MX_1 & = m = 0.5\\
    \sigma & = D\xi = 1.65 & DX_1 & = \sigma^2 = 2.7225\\
    & & Q & = 1.0
\end{align*}

Для описанного процесса были проведны 100 экспериментальных прогонов модели ветвящегося процесса с целью определения
вероятностей $P(X_n=k)$ для $k=0,1,\dots,8$ при $n\rightarrow\infty$.
Кроме того для каждого из значений были подсчитаны доверительные интервалы на уровне значимости 5\%.
Вероятности с соответствующими доверительными интервалами приведены ниже.

\begin{align*}
P(X_n = 0) & = 0.47311, & 0.45808 & < P(X_n = 0) < 0.48814\\
P(X_n = 1) & = 0.47311, & 0.45808 & < P(X_n = 1) < 0.48814\\
P(X_n = 2) & = 0.01361, & 0.00135 & < P(X_n = 2) < 0.02587\\
P(X_n = 3) & = 0.007, & -0.00086 & < P(X_n = 3) < 0.01486\\
P(X_n = 4) & = 0.01478, & 0.00158 & < P(X_n = 4) < 0.02798\\
P(X_n = 5) & = 0.005, & -0.00187 & < P(X_n = 5) < 0.01187\\
P(X_n = 6) & = 0.00667, & -0.0025 & < P(X_n = 6) < 0.01583\\
P(X_n = 7) & = 0.0, & 0.0 & < P(X_n = 7) < 0.0\\
P(X_n = 8) & = 0.0025, & -0.00239 & < P(X_n = 8) < 0.00739
\end{align*}

\subsection*{Критический процесс}

Условием критического процесса является равенство $m = 1$, которое достигается, например, следующим распределением вероятности:
\begin{align*}
    p_0 & = 0.6 & p_2 & = 0.3 & p_3 & = 0.04 & p_4 & = 0.03 & p_5 & = 0.02 & p_6 & = 0.01
\end{align*}

Параметры приведенного выше распределения следующие:
\begin{align*}
    m & = M\xi = 1 & MX_1 & = m = 1\\
    \sigma & = D\xi = 1.9 & DX_1 & = \sigma^2 = 3.61\\
    & & Q & = 1.0
\end{align*}

Для описанного процесса были проведны 100 экспериментальных прогонов модели ветвящегося процесса с целью определения
вероятностей $P(X_n=k)$ для $k=0,1,\dots,8$ при $n\rightarrow\infty$.
Кроме того для каждого из значений были подсчитаны доверительные интервалы на уровне значимости 5\%.
Вероятности с соответствующими доверительными интервалами приведены ниже.

\begin{align*}
P(X_n = 0) & = 0.3829, &  0.3479  & < P(X_n = 0) < 0.41789\\
P(X_n = 1) & = 0.3949, &  0.36471 & < P(X_n = 1) < 0.42508\\
P(X_n = 2) & = 0.09993, &  0.06552 & < P(X_n = 2) < 0.13434\\
P(X_n = 3) & = 0.01243, &  0.00343 & < P(X_n = 3) < 0.02143\\
P(X_n = 4) & = 0.02229, &  0.00717 & < P(X_n = 4) < 0.0374\\
P(X_n = 5) & = 0.01547, &  0.00665 & < P(X_n = 5) < 0.02429\\
P(X_n = 6) & = 0.02087, &  0.00692 & < P(X_n = 6) < 0.03482\\
P(X_n = 7) & = 0.00811, &  0.00203 & < P(X_n = 7) < 0.01419\\
P(X_n = 8) & = 0.006, &  0.00134 & < P(X_n = 8) < 0.01066
\end{align*}

\subsection*{Надкритический процесс}

Условием надкритического процесса является отношение $m > 1$, которому соответствует, например, следующее распределение вероятности:
\begin{align*}
    p_0 & = 0.65 & p_2 & = 0.11 & p_3 & = 0.09 & p_4 & = 0.07 & p_5 & = 0.05 & p_6 & = 0.03
\end{align*}

Параметры приведенного выше распределения следующие:
\begin{align*}
    m & = M\xi = 1.2 & MX_1 & = m = 1.2\\
    \sigma & = D\xi = 3.26 & DX_1 & = \sigma^2 = 10.6276\\
    & & Q & = 0.87304
\end{align*}

Для описанного процесса были проведны 100 экспериментальных прогонов модели ветвящегося процесса с целью определения
вероятностей $P(X_n=k)$ для $k=0,1,\dots,8$ при $n\rightarrow\infty$.
Кроме того для каждого из значений были подсчитаны доверительные интервалы на уровне значимости 5\%.
Вероятности с соответствующими доверительными интервалами приведены ниже.

\begin{align*}
P(X_n = 0) & = 0.37043, & 0.33175 & < P(X_n = 0) < 0.40911\\
P(X_n = 1) & = 0.38943, & 0.35761 & < P(X_n = 1) < 0.42125\\
P(X_n = 2) & = 0.0211, & 0.00528 & < P(X_n = 2) < 0.03691\\
P(X_n = 3) & = 0.02617, & 0.01038 & < P(X_n = 3) < 0.04196\\
P(X_n = 4) & = 0.0315, & 0.01523 & < P(X_n = 4) < 0.04777\\
P(X_n = 5) & = 0.01817, & 0.00511 & < P(X_n = 5) < 0.03122\\
P(X_n = 6) & = 0.01893, & 0.00937 & < P(X_n = 6) < 0.02849\\
P(X_n = 7) & = 0.0175, & 0.00569 & < P(X_n = 7) < 0.02931\\
P(X_n = 8) & = 0.00743, & 0.00205 & < P(X_n = 8) < 0.0128
\end{align*}

\section*{Выводы}

В рамках настоящей лаборатной работы были исследованы модели ветвящегося процесса, заданного производящей функцией.
Во-первых, были подобраны распределения вероятностей, соответствующие докритическому, критическому и надкритическому процессам.
Во-вторых для каждого из процессов было моделирование с целью определения вероятности вырождения ветвящегося процесса,
а также вероятностей $P(X_n=k)$ для заданных $k$ и $n\rightarrow\infty$.

В результате, для докритического и критического процессов вероятность вырождения $Q$ получилась равной 1, что означает,
что ветвящийся процесс вырождается всегда. Определенные вероятности $P(X_n=k)$ согласуются с данным утверждением, обладая
крайне высокими значениями вероятности $~47\%$ для случаев с $k=0$ и $k=1$.
Для случая надкритического процесса $Q$ получилась равной 0.87304, означающей, что ветвящийся процесс не вырождается с
ненулевой вероятностью. В ходе проведения моделирования надкритического процесса в ряде случаев наблюдались процессы,
в которых количество особей увеличивалось в сотни раз крайне быстро по сравнению с большей частью процессов, в которых
количество особей не превышало в общей сложности 10-ти.

\section*{Листинги}

\subsection*{Листинг основного скрипта}
\lstinputlisting[language=Python,texcl=true]{\jobname/lab.py}

\subsection*{Листинг скрипта, содержащего модель ветвящегося процесса}
\lstinputlisting[language=Python,texcl=true]{\jobname/branching.py}

\subsection*{Листинг скрипта, содержащего генераторы}
\lstinputlisting[language=Python,texcl=true]{\jobname/../common/gen.py}

\subsection*{Листинг скрипта, содержащего утилиты}
\lstinputlisting[language=Python,texcl=true]{\jobname/../common/utils.py}

\subsection*{Листинг скрипта, инициализируещего логирование}
\lstinputlisting[language=Python,texcl=true]{\jobname/../common/log.py}
